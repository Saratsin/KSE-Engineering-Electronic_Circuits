\newcommand{\marginnote}[1]{\marginpar{\framebox{\framebox{#1}}}}
% Comment macro: Usage \comment{Author}{Comment}
% \newcommand{\comment}[2]{[\marginnote{#1}\textit{#1}:\ \textit{#2}]}

% \newcommand{\bra}[1]{#1^H}
% \newcommand{\ket}[1]{#1}
% \newcommand{\innerp}[2]{#1^H#2}
% \newcommand{\outerp}[2]{#1#2^H}
% \newcommand{\matrixel}[3]{#1^H#2#3}
% \newcommand{\proj}[1]{\outerp{#1}{#1}}
% \newcommand{\abs}[1]{\vert#1\vert}
% 
% 
% %norms
% \newcommand{\norm}[2]{\left\Vert#1\right\Vert_{#2}}
% 
% %a matrix or vector
% \newcommand{\mat}[2]{\lefto[\begin{array}{#1}#2\end{array}\right]}
% 
% %mathbf etc.
% \newcommand{\mbf}{\mathbf}
% \newcommand{\mbb}{\mathbb}
% \newcommand{\mcl}{\mathcal}
 \newcommand{\trm}{\textrm}
% 
% 
% %Theorems, etc.
% %\theoremstyle{definition}
% \theoremstyle{plain}
%
% 
% %Expectation
% \newcommand{\expect}[1]{\langle#1\rangle}
% 
% %Trace
% \DeclareMathOperator{\tr}{tr}
% \DeclareMathOperator{\Tr}{Tr}
% 
% %Miscellaneous stuff
% \DeclareMathOperator{\supp}{supp}
% \DeclareMathOperator{\Span}{span}
% 
% %identity
% \DeclareMathOperator{\id}{\mathbf{1}}
% 
% %BS addition
% \DeclareMathOperator{\bst}{\stackrel{\land}{+}}
% \DeclareMathOperator{\bs}{\widehat{+}}
% 
% %ladder operators
% \newcommand{\up}{a^\dagger}
% \newcommand{\down}{a}
% 
% %vec operator
% %\DeclareMathOperator{\vect}{vec}
% 
% %equations
% \def\ba#1\ea{\begin{align*}#1\end{align*}}
% %with numeration
% \def\ban#1\ean{\begin{align}#1\end{align}}
% % the same, centered
% \def\bac#1\eac{\vspace{\abovedisplayskip}{\par\centering$\begin{aligned}#1\end{aligned}$\par}\addvspace{\belowdisplayskip}}

%%%%% parentheses denoting arguments, we want an opening object. Use \lefto in
%%%%% these cases.
% \newcommand{\lefto}{\mathopen{}\left}
% 
% %real and imaginary part
%\renewcommand{\Re}{\mathfrak{Re}}
%\renewcommand{\Im}{\mathfrak{Im}}
% 
%\renewcommand{\Re}{\text{R}}
%\renewcommand{\Im}{\text{I}}
% 
% % in case you need to break equations over several lines
 \newcommand{\dummyrel}[1]{\mathrel{\hphantom{#1}}\strut\mskip-\medmuskip}
%%%%%%%%%%%%%%%%%%%%%
%:- Theorems
\newtheorem{thm}{Theorem}
\newtheorem{cor}[thm]{Corollary}   % Turned off theorem numbering
\newtheorem{prop}{Proposition}
\newtheorem{defi}{Definition}
\newtheorem{lem}[thm]{Lemma} % Turned off lemma numbering
\newtheorem{exmpl}{Example}
\newtheorem{obs}{Observation}
\newtheorem{rem}{Remark}
\newtheorem{asm}{Assumption}
% \newtheorem{st}{Statement}
% \newtheorem{conj}{Conjecture}

%%%%%
%% robust recovery from sparse noise
\safemath{\dictab}{[\,\dicta\,\,\dictb\,]}


\safemath{\ysig}{\bmy}
\safemath{\ysighat}{\hat{\ysig}}
\safemath{\ysigdim}{M}
%
\safemath{\xsig}{\bmx}
\safemath{\xsigdim}{N}
\safemath{\nx}{n_x}
%
\safemath{\zsig}{\bmz}
\safemath{\zsigdim}{\ysigdim}
%
\safemath{\rsig}{\bmr}
%
\safemath{\Adict}{\bA}
\safemath{\Adicttilde}{\widetilde{\Adict}}
\safemath{\Adictdim}{\outputdim\times\xsigdim}
\safemath{\avec}{\bma}
\safemath{\avectilde}{\tilde{\avec}}
%
\safemath{\Bdict}{\bB}
\safemath{\Bdicttilde}{\widetilde{\Bdict}}
%
\safemath{\Cdict}{\bC}
\safemath{\cvec}{\bmc}
%
\safemath{\Ddict}{\bD}
\safemath{\Ddictdim}{\ysigdim\times\xsigdim}
\safemath{\dvec}{\bmd}
\safemath{\Ddicttilde}{\widetilde{\bD}}
%
\safemath{\Bonb}{\bB}
\safemath{\bvec}{\bmb}
\safemath{\Bonbdim}{\ysigdim\times\ysigdim}
%
\safemath{\noise}{\bmn}
\safemath{\noisedim}{\ysigim}
%
\safemath{\err}{\bme}
\safemath{\errdim}{\ysigdim}
\safemath{\errset}{\setE}
\safemath{\nerr}{n_e}
%
\safemath{\delop}{\bP_\errset}
\safemath{\delopc}{\bP_{{\errset}^c}}

%




%%
% Complex i and j 
\safemath{\cplxi}{\imath}
\safemath{\cplxj}{\jmath}
% Comb signal
%\safemath{\comb}{\matI\matI\matI}
\newcommand{\comb}[1]{\vecdelta_{#1}}
%:- Definition dictionary
\safemath{\dict}{\matD}
\safemath{\inputdim}{N}		% number of columns of dictionary D
\safemath{\outputdim}{M}		%number of rows of dictionary D
\safemath{\sparsity}{S}	%sparsity
\safemath{\inputdimA}{{N_a}}	%total number of elements in dictionary A
\safemath{\inputdimB}{{N_b}}	%total number of elements in dictionary B
\safemath{\elemA}{{n_a}}	%number of elements chosen from dictionary A
\safemath{\elemB}{{n_b}}	%number of elements chosen from dictionary B
\safemath{\resA}{\matR_a}	%restriction map to elements of dictionary A
\safemath{\resB}{\matR_b}	%restriction map to elements of dictionary B
\safemath{\subD}{\matS} %subdictionary
\safemath{\subA}{\matS_a} %subdictionary part of A
\safemath{\subB}{\matS_b} %subdictionary part of B
\safemath{\dicta}{\matA} 	% first subdictionary
\safemath{\dictb}{\matB} 	% second subdictionary
\safemath{\hollowS}{H}
\safemath{\hollowA}{H_a}
\safemath{\hollowB}{H_b}
\safemath{\cross}{Z}
\safemath{\coh}{\mu_d}			% coherence dictionary
\safemath{\coha}{\mu_a}			% coherence first subdictionary
\safemath{\cohb}{\mu_b}			% coherence second subdictionary
\safemath{\mubs}{\nu}	%block sub-coherence
\safemath{\cohm}{\mu_m} %mutual coherence
\safemath{\dictset}{\setD}	% set of dictionaries
\safemath{\dictsetp}{\dictset(\coh,\coha,\cohb)}	% set of dictionaries parametrized
\safemath{\dictsetgen}{\dictset_\text{gen}}
\safemath{\dictsetgenp}{\dictsetgen(\coh)}
\safemath{\dictsetonb}{\dictset_\text{onb}}
\safemath{\dictsetonbp}{\dictsetonb(\coh)}

\safemath{\leftside}{U}
\safemath{\rightsideA}{R_a}
\safemath{\rightsideB}{R_b}

\safemath{\indexS}{\setI_S} %set of indices participating in sub-dictionary S


\safemath{\na}{n_a}			% cardinality of set of linearly independent columns of first dictionary
\safemath{\nb}{n_b}			% cardinality of set of linearly independent columns of second dictionary
\safemath{\coeffa}{p_i}	%coefficients for columns of A
\safemath{\coeffb}{q_j}	%coefficients for columns of B
\safemath{\seta}{\setP}		% set of linearly independent columns of A
\safemath{\setb}{\setQ}     % set of linearly independent columns of B
\safemath{\setw}{\setW}	%set of n largest elements of w
\safemath{\setz}{\setZ}	%set of L-n largest elements of z
\safemath{\cola}{\veca}		% generic element of the dictionary A
\safemath{\colb}{\vecb}		% generic element of the dictionary B
\safemath{\cold}{\vecd}		% generic element of the dictionary D
\safemath{\inputvec}{\vecx} 	%coefficient vector (input)
\safemath{\error}{\vece}	%error vector
\safemath{\noiseout}{\vecz} 	%noisy output vector
\safemath{\inputvecel}{x}
\safemath{\inputveca}{\vecx_a}
\safemath{\inputvecb}{\vecx_b}
\safemath{\outputvec}{\vecy}	%output of Dictionary
\safemath{\lambdamin}{\lambda_{\mathrm{min}}}
%:- Math operators
\DeclareMathOperator{\spark}{spark}
\newcommand{\pos}[1]{\lefto[#1\right]^+}
\newcommand{\normtwo}[1]{\vecnorm{#1}_2}
\newcommand{\normone}[1]{\vecnorm{#1}_1}
\newcommand{\normzero}[1]{\vecnorm{#1}_0}
\newcommand{\norminf}[1]{\vecnorm{#1}_\infty}
\newcommand{\norminftilde}[1]{\vecnorm{#1}_{\widetilde\infty}}
\newcommand{\spectralnorm}[1]{\vecnorm{#1}_{2,2}}
%\newcommand{\spectralnorm}[1]{\vecnorm{#1}}
\safemath{\elltwo}{\ell_2}
\safemath{\ellone}{\ell_1}
\safemath{\ellzero}{\ell_0}
\safemath{\ellinf}{\ell_\infty}
\safemath{\ellinftilde}{\ell_{\widetilde\infty}}
\safemath{\licard}{Z(\coh,\coha,\cohb)}
\safemath{\xsol}{\hat{x}}
\safemath{\xbord}{x_b}		%Solution at the border
\safemath{\xstat}{x_s}		%Solution stationary in l0 prob
\safemath{\xstatLone}{\tilde{x}_s}
\safemath{\order}{\mathcal{O}} %order notation (big O)
\safemath{\scales}{\Theta} %scales as
%
\safemath{\ones}{\mathbf{1}} %all ones matrix
\safemath{\zeroes}{\mathbf{0}} %all zeroes matrix
%
\safemath{\thlone}{\kappa(\coh,\cohb)} %treshold l1 problem
\safemath{\constoneA}{\delta} %constant in l1 theorem to save space
\safemath{\constoneB}{\epsilon} %constant in l1 theorem to save space
\safemath{\nlarge}{L}				   %num large elements
\safemath{\sumlarge}{S_\nlarge}
\DeclareMathOperator{\kernel}{kern}	   % kernel of a matrix
\safemath{\maxlarger}{P_\nlarge}	   % maximum in Gribonval and Nielsen
\safemath{\Pzero}{\textrm{P0}}	
\safemath{\Pone}{\textrm{P1}}
\safemath{\vecfir}{\vecw}			 % \vecv element of the kernel of the dictionary, \vecv=[\vecfir \vecsec]
\safemath{\vecsec}{\vecz}
\safemath{\elvecfir}{w}              % element of vecfir
\safemath{\elvecsec}{z}				 % element of vecsec
\safemath{\nlargefir}{n}
\safemath{\normout}{\gamma}
\safemath{\auxfun}{h}
\safemath{\supp}{\textrm{supp}}%support

\safemath{\indexa}{\ell}
\safemath{\indexb}{r}
\safemath{\indexc}{i}
\safemath{\indexd}{j}


\safemath{\project}{P}%projector

% Current season & year macros:
% Define a custom command for the season and year
\newcommand{\currentSeasonAndYear}{%
    \begingroup
    \month=\the\month
    \year=\the\year
    \ifnum\month=12
        Winter \the\year%
    \else
        \ifnum\month<3
            Winter \the\year%
        \else
            \ifnum\month<6
                Spring \the\year%
            \else
                \ifnum\month<9
                    Summer \the\year%
                \else
                    \ifnum\month<12
                        Fall \the\year%
                    \else
                        Winter \the\year%
                    \fi
                \fi
            \fi
        \fi
    \fi
    \endgroup
}

% Code insertion macros:

\newcommand{\added}[1]{\colorbox{green!20}{#1}}
\newcommand{\deleted}[1]{\colorbox{red!20}{\st{\texttt{#1}}}}